\chapter{Аналитическая часть}
\section{Преломление лучей}

Когда свет проходит из одной среды в другую, например из воздушной в водную и т.п., луч отклоняется, такое явление называется преломление. Его пример видно на рисунке~\ref{img:pencil-glass}, из-за отклонения лучей света наблюдателю кажется, что объект разрывается.

\includeimage
{pencil-glass} % Имя файла без расширения (файл должен быть расположен в директории inc/img/)
{f} % Обтекание (без обтекания)
{h} % Положение рисунка (см. figure из пакета float)
{0.8\textwidth} % Ширина рисунка
{Карандаш в стакане с водой, преломление света нарушает визуальную непрерывность.} % Подпись рисунка

Закон описывающий направления отклонения, изображенном на  рисунке~\ref{img:snells-law}, называется закон Снелиуса~\cite{rogers-book}:

\begin{equation}
	\frac{\sin\theta_1}{\sin\theta_2} = \frac{n_2}{n_1},
\end{equation}

где:

\begin{itemize}
	\item \(\theta_1\), \(\theta_2\) --- угол падения и преломленного угла луча,
	\item \(n_1\), $n_2$ --- коэффициенты плотности среды,
	\item \(v_1\), $v_2$ --- соответствующие скорости распространения света в средах,
	\item \textit{normal} --- нормаль к поверхности \(N\),
	\item $P$ --- вектор, описывающий направление падающего луча в пространстве,
	\item $Q$ --- исходящего.
\end{itemize}

\includeimage
{snells-law} % Имя файла без расширения (файл должен быть расположен в директории inc/img/)
{f} % Обтекание (без обтекания)
{h} % Положение рисунка (см. figure из пакета float)
{0.2\textwidth} % Ширина рисунка
{Преломление одного луча.} % Подпись рисунка

Важное замечание -- если луч падает их более оптически плотной среде ($n_1 > n_2$), под определенным углом (критическим) явление преломления выродится в случай внутреннего отражения. И значение угла $\theta_1$ находится при $\theta_2 = \pi / 2$, как:

\begin{equation}
	\sin\theta_1 = \frac{n_2}{n_1} \cdot \sin\theta_2 = \frac{n_2}{n_1},
\end{equation}

и угол исходящего луча $\theta_2$ будет равен $\theta_1$.

Направление исходящего луча также вычисляется следующим выражением:
\begin{equation}
	 Q =r \cdot P + \left[r \cdot c- {\sqrt {1-r^{2} \cdot (1-c^{2})}}\right] \cdot N,
\end{equation}

где $r=n_{1}/n_{2}$ и $c = -\cos\theta_1 = -N \cdot P$.

Если луч падает под критическим углом, то

\begin{equation}
	r^2 \cdot (1 - c^2) = r^2 \cdot \sin^2 \theta_1 = \frac{n_1^2}{n_2^2} \cdot \frac{n_2^2}{n_1^2} = 1, 
\end{equation}

то есть подкоренное меньше нуля, если происходит внутреннее отражение.

Направление отраженного луча определяется, как:

\begin{equation}
	Q = P - 2 \cdot (N \cdot L) \cdot N = P + 2 \cdot c \cdot N
\end{equation}

\section{Задача}

Формализованная задача представлена в виде диаграммы~на~рисунке~\ref{img:Task}.\newline

\includeimage{Task}{f}{h}{0.8\textwidth}{IDEF0 диаграмма формализованной задачи}


\section{Объекты}

Сцена содержит в себе следующие объекты:

\begin{itemize}
	\item модели --- видимые объемные фигуры произвольной формы с параметрами:
	\begin{enumerate}
		\item цвет поверхности,
		\item прозрачность,
		\item коэффициент преломления (оптическая плотность среды);
	\end{enumerate}
	\item источники света, с параметрами: 
	\begin{enumerate}
		\item цвет,
		\item интенсивность излучения;
	\end{enumerate}
	\item точки обзора (камеры) --- являются наблюдателями на сцене, всегда присутствует хотя бы одна камера, с параметрами:
	\begin{enumerate}
		\item разрешение (в том числе задает соотношение сторон изображения),
		\item угол обзора.
	\end{enumerate}
\end{itemize}

Каждый объект обладает положением в пространстве


\subsection{Описание}
Для представления объемных сущностей в компьтерной графике ниже представлены~\cite[с.с.~341--350]{cg-priciples}:
\begin{enumerate}
	\item аналитическая --- в виде уравнения поверхности, позволяет описать примитивную геометрию объекта, например прямой, плоскости, шара, тора и т.п., в пространстве;
	\item воксельная --- использует блоки для построения;
	\item полигональная --- в виде многогранника, позволяет описывать сложную геометрию.
\end{enumerate}

\subsection{Представление}

Для решения задачи необходимо выбрать то представление объекта, которого будет достаточно для представления любого на сцене. Так как объекты могут быть любой формы, и описываются сущности имеющие объем, выбран полигональное представление.


\section{Алгоритмы удаления невидимых поверхностей}

Для удаления невидимых поверхностей в большинстве случаев используют следующие алгоритмы:

\begin{itemize}
	\item алгоритм Робертса,
	\item алгоритм Художника,
	\item алгоритм, использующий Z-буфер,
	\item алгоритм обратной трассировки лучей.
\end{itemize}

\subsection{Алгоритм Робертса}

Алгоритм основан на анализе нормалей граней и определении их ориентации относительно наблюдателя.

Основные этапы алгоритма:
\begin{enumerate}
	\item 
	\item 
	\item
\end{enumerate}

Преимущества

\begin{itemize}
	\item 
	\item 
	\item
\end{itemize}

Недостатки

\begin{itemize}
	\item 
	\item 
	\item
\end{itemize}

Таким образом

\subsection{Алгоритм Художника}

Основные этапы алгоритма:
\begin{enumerate}
	\item 
	\item 
	\item
\end{enumerate}

Преимущества

\begin{itemize}
	\item 
	\item 
	\item
\end{itemize}

Недостатки

\begin{itemize}
	\item 
	\item 
	\item
\end{itemize}

Таким образом

\subsection{Алгоритм, использующий Z-буфер}

Основные этапы алгоритма:
\begin{enumerate}
	\item 
	\item 
	\item
\end{enumerate}

Преимущества

\begin{itemize}
	\item 
	\item 
	\item
\end{itemize}

Недостатки

\begin{itemize}
	\item 
	\item 
	\item
\end{itemize}

Таким образом

\subsection{Алгоритм обратной трассировки лучей}

Основные этапы алгоритма:
\begin{enumerate}
	\item 
	\item 
	\item
\end{enumerate}

Преимущества

\begin{itemize}
	\item 
	\item 
	\item
\end{itemize}

Недостатки

\begin{itemize}
	\item 
	\item 
	\item
\end{itemize}

Таким образом

\subsection{Выбранный алгоритм}

Так как все описанные алгоритмы, кроме обратной трассировки,не учитывают искожений

\section{Освещение}

Модели освещения рассматривают 2 вида: локальная и глобальная.

\subsection{Модели освещения}

Локальная модель освещения учитывает модели индивидуально, что не подходит для реализации задачи, так как именно другие объекты будут видны в результате преломления. Что учитывает глобальная~\cite[с.~464,~с.~502,~с.~548]{rogers-book}.

Уравнение интенсивности:
\begin{equation}
	I = k_{\alpha} \cdot I_{\alpha} + k_d \cdot \sum_{i=1}^{N} (I_{L_i} \cdot \vec{n} \cdot \vec{L_i}) + k_s \cdot \sum_{i=1}^{N} (I_{L_i} \cdot (\vec{S} \cdot \vec{L_i}) ^ n) + k_s \cdot I_s + k_t \cdot I_t,
\end{equation}

где коэффициенты
\begin{itemize}
	\item $k_{\alpha}$ --- фоновый,
	\item $k_d$ --- диффузный,
	\item $k_s$ --- зеркальный,
	\item $k_t$ --- преломленного луча,
	\item $n$ --- аппроксимирующий распределение лучей от отражения,
\end{itemize}

и вектора
\begin{itemize}
	\item $\vec{L}$ --- направления до точечного источника света,
	\item $\vec{S}$ --- направление до наблюдателя,
	\item $\vec{n}$ --- нормали к поверхности.
\end{itemize}

\subsection{Тени}

Для определения интенсивности затененных областей удобно использовать алгоритм трассировки лучей, для это испускаются зондирующие лучи до источников света, если пересечение установлено, то область находится в тени~\cite[с.~517]{rogers-book}. Однако, когда рассматриваются прозрачные модели, точка может не быть затененной, поэтому для каждого зонда и источника коэффициент затенения от одного объекта составляет~\cite[с.~368]{cg-priciples}:

\begin{equation}
	k_{shadow,i} = k_{t, i}
\end{equation} 

Если луч пересекает несколько объектов, то коэффициент для $N_O$ полных пересечений моделей на отрезке до источника света будет:
\begin{equation}
	k_{shadow} = \prod_{i=1}^{N_O} k_{t, i}
\end{equation}

\subsection{Полная интенсивность}

Итоговая интенсивности в точке вычисляется по формуле~\ref{fn:intensity}.
\begin{equation} \label{fn:intensity}
		I = k_{\alpha} \cdot I_{\alpha} + 
		\sum_{i=1}^{N_L} \left[
			\prod_{j=1}^{N_O}	k_{t, j} \cdot 
			(k_d \cdot \vec{n} \cdot \vec{L_i} + k_s \cdot (\vec{S} \cdot \vec{L_i}) ^ n) 
		\right] \cdot I_{L_i} + 
		k_s \cdot I_s + 
		k_t \cdot I_t
\end{equation}

\section{Выводы из аналитической части}

\clearpage
