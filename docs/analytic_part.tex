\chapter{Аналитическая часть}
\section{Преломление лучей}

Когда свет проходит из одной среды в другую, например из воздушной в водную и т.п., луч отклоняется, такое явление называется преломление. Его пример видно на рисунке~\ref{img:pencil-glass}, из-за отклонения лучей света наблюдателю кажется, что объект разрывается.

\includeimage
{pencil-glass} % Имя файла без расширения (файл должен быть расположен в директории inc/img/)
{f} % Обтекание (без обтекания)
{h} % Положение рисунка (см. figure из пакета float)
{0.8\textwidth} % Ширина рисунка
{Карандаш в стакане с водой, преломление света нарушает визуальную непрерывность.} % Подпись рисунка

Закон описывающий направления отклонения, изображенном на  рисунке~\ref{img:snells-law}, называется закон Снелиуса~\cite{rogers-book}:

\begin{equation}
	\frac{\sin\theta_1}{\sin\theta_2} = \frac{n_2}{n_1},
\end{equation}

где:

\begin{itemize}
	\item \(\theta_1\), \(\theta_2\) --- угол падения и преломленного угла луча,
	\item \(n_1\), $n_2$ --- коэффициенты плотности среды,
	\item \(v_1\), $v_2$ --- соответствующие скорости распространения света в средах,
	\item \textit{normal} --- нормаль к поверхности \(N\),
	\item $P$ --- вектор, описывающий направление падающего луча в пространстве,
	\item $Q$ --- исходящего.
\end{itemize}

\includeimage
{snells-law} % Имя файла без расширения (файл должен быть расположен в директории inc/img/)
{f} % Обтекание (без обтекания)
{h} % Положение рисунка (см. figure из пакета float)
{0.2\textwidth} % Ширина рисунка
{Преломление одного луча.} % Подпись рисунка

Важное замечание -- если луч падает их более оптически плотной среде ($n_1 > n_2$), под определенным углом (критическим) явление преломления выродится в случай внутреннего отражения. И значение угла $\theta_1$ находится при $\theta_2 = \pi / 2$, как:

\begin{equation}
	\sin\theta_1 = \frac{n_2}{n_1} \cdot \sin\theta_2 = \frac{n_2}{n_1},
\end{equation}

и угол исходящего луча $\theta_2$ будет равен $\theta_1$.

Направление исходящего луча также вычисляется следующим выражением:
\begin{equation}
	 Q =r \cdot P + \left[r \cdot c- {\sqrt {1-r^{2} \cdot (1-c^{2})}}\right] \cdot N,
\end{equation}

где $r=n_{1}/n_{2}$ и $c = -\cos\theta_1 = -N \cdot P$.

Если луч падает под критическим углом, то

\begin{equation}
	r^2 \cdot (1 - c^2) = r^2 \cdot \sin^2 \theta_1 = \frac{n_1^2}{n_2^2} \cdot \frac{n_2^2}{n_1^2} = 1, 
\end{equation}

то есть подкоренное меньше нуля, если происходит внутреннее отражение.

Направление отраженного луча определяется, как:

\begin{equation}
	Q = P - 2 \cdot (N \cdot L) \cdot N = P + 2 \cdot c \cdot N
\end{equation}

\section{Задача}

Формализованная задача представлена в виде диаграммы~на~рисунке~\ref{img:CG-idef0}.\newline

\includeimage{CG-idef0}{f}{h}{0.8\textwidth}{IDEF0 диаграмма формализованной задачи.}


\section{Объекты}

Сцена содержит в себе следующие объекты:

\begin{itemize}
	\item модели --- видимые объемные фигуры произвольной формы с параметрами:
	\begin{enumerate}
		\item цвет поверхности,
		\item зеркальность,
		\item коэффициент рассеивания,
		\item прозрачность,
		\item коэффициент преломления (оптическая плотность среды)
		\item положение в пространстве (глобальные координаты, поворот, масштаб);
	\end{enumerate}
	\item источники света, с параметрами: 
	\begin{enumerate}
		\item цвет,
		\item интенсивность излучения,
		\item \item положение в пространстве (глобальные координаты, поворот, масштаб);
	\end{enumerate}
	\item точки обзора (камеры) --- являются наблюдателями на сцене, всегда присутствует хотя бы одна камера, с параметрами:
	\begin{enumerate}
		\item разрешение (в том числе задает соотношение сторон изображения),
		\item угол обзора,
		\item положение в пространстве (глобальные координаты, поворот, масштаб).
	\end{enumerate}
\end{itemize}


\subsection{Описание}
Для представления объемных сущностей в компьютерной графике ниже представлены~\cite[с.с.~341--350]{cg-priciples}:
\begin{enumerate}
	\item аналитическая --- в виде уравнения поверхности, позволяет описать примитивную геометрию объекта, например прямой, плоскости, шара, тора и т.п., в пространстве;
	\item воксельная --- использует блоки для построения;
	\item полигональная --- в виде многогранника, позволяет описывать сложную геометрию.
\end{enumerate}

\subsection{Представление}

Для решения задачи необходимо выбрать то представление объекта, которого будет достаточно для представления любого на сцене. Так как объекты могут быть любой формы, и описываются сущности имеющие объем, выбран полигональное представление.


\section{Алгоритмы удаления невидимых поверхностей}

Для корректного изображения сцены, нужно удалить невидимые поверхности, так как задача --- генерация изображений с учетом преломления света, а значит предполагается, что некоторые объекты пропускают свет, таким образом геометрически объект, расположенный за прозрачной материей, не будет доступен, однако визуально 2-й объект будет виден. Кроме того, за счет преломления света на кадре могут быть 


Для удаления невидимых поверхностей в большинстве случаев используют следующие алгоритмы:

\begin{itemize}
	\item алгоритм Робертса~\cite[с. 303]{rogers-book},
	\item алгоритм Художника~\cite[с. 387]{rogers-book},
	\item алгоритм, использующий Z-буфер~\cite[с. 375]{rogers-book},
	\item алгоритм обратной трассировки лучей~\cite[с. 432]{rogers-book}.
\end{itemize}

\subsection{Алгоритм Робертса}

Алгоритм основан на анализе нормалей граней и определении их ориентации относительно наблюдателя. Для этого он использует матрицы тела $T$ размером $M \times 4$, где $M$ -- количество поверхностей объекта.

Основные этапы алгоритма:
\begin{enumerate}
	\item удаления граней, экранируемых самим телом;
	\item удаление частей, экранируемых другими объектами;
	\item устранение отрезков, пересекающих плоскости.
\end{enumerate}

Преимущества:

\begin{itemize}
	\item не требует создания буферов размером пропорциональному размеру экрана,
	\item не использует растеризованные формы объектов,
	\item эффективен для выпуклых объектов без самопересечений.
\end{itemize}

Недостатки:

\begin{itemize}
	\item предназначен для работы с выпуклыми многогранниками;
	\item при преломлении света нарушается базовое предположение о прямолинейном распространении лучей;
	\item неприменим к прозрачным и полупрозрачным материалам, где <<невидимые>> грани влияют на итоговое изображение через преломление.
\end{itemize}

Классический алгоритм Робертса критически ограничен при работе с преломляющими материалами, поскольку требует модификации для трассировки преломлённых лучей и учёта вклада всех поверхностей в формирование изображения, а не только "видимых" в геометрическом смысле. Поэтому этот алгоритм не подходит для решения этой задачи.

\subsection{Алгоритм Художника}

Основные этапы алгоритма:
\begin{enumerate}
	\item вычисление глубины каждого полигона,
	\item сортировка этих полигонов по глубине,
	\item проверка перекрытий и разрешение конфликтов,
	\item последовательная отрисовка полигонов от дальних к ближним по списку.
\end{enumerate}

Преимущества:

\begin{itemize}
	\item возможность реализации прозрачности,
	\item не требует дополнительного буфера.
\end{itemize}

Недостатки:

\begin{itemize}
	\item требует предварительной сортировки,
	\item сложности обработки перекрывающихся многоугольников,
	\item при изменении положения наблюдателя требует повторной сортировки.
\end{itemize}

Алгоритм не подходит для визуализации отраженных и преломленных объектов, поэтому он не подходит для решения поставленной задачи.

\subsection{Алгоритм, использующий Z-буфер}

Алгоритм, работающий в пространстве изображения, использует буфер глубин каждого пикселя буфера кадра, в результате чего на изображении будут только видимые поверхности.

Основные этапы алгоритма:
\begin{enumerate}
	\item заполнить буфер кадра $b_{frame}$ фоновым значением,
	\item заполнить Z-буфер $b_{z}$ минимальным значением,
	\item растеризировать все объекты
	\item для каждого пикселя $point$ вычислить его глубину $z(point)$, если $b_z(point) < z(point)$, то присвоить $b_z(point)$ значение $z(point)$ и $b_{frame}(point)$ значение цвета поверхности.
\end{enumerate}

Преимущества:

\begin{itemize}
	\item подходит для обработки разных представлений тел,
	\item не требует предварительной сортировки списка объектов сцены,
	\item обладает линейной сложностью,
	\item не случается конфликтных ситуаций.
\end{itemize}

Недостатки:

\begin{itemize}
	\item ошибка точности, связанная с перспективным преобразованием,
	\item предполагает прямолинейное распространение света,
	\item не позволяет реализовать прозрачность,
	\item слишком прост в реализации, чтобы визуализировать сложные сцены с большим количеством зависимостей.
\end{itemize}

Алгоритм использующий Z-буфер, не позволяет решить поставленную задачу, а значит --- не подходит.

\subsection{Алгоритм обратной трассировки лучей}

Алгоритм <<грубой силы>>, наиболее приближен к физическим моделям распространения света, потому что просчитывается каждый луч. 
Изначально рассматривались 2 вида трассировки: прямая -- от источника света до наблюдателя и обратная. 
Прямое трассирование луча является менее эффективным способом решения задачи, так как большая часть выпущенных из источника лучей не дойдут до наблюдателя. 
Второй вид, наоборот, позволяет вычислять только луч, влияющие на результат.

В настоящее время разработано множество модификаций, преследующие различные цели. За счет того, что просчитывается каждый луч, доступна реализация различных оптических явлений, в том числе преломления и прозрачности.


Основные этапы алгоритма:
\begin{enumerate}
	\item для каждого пикселя экрана сформировать луч,
	\item найти ближайшее пересечение луча с моделью на сцене.
\end{enumerate}

Преимущества:

\begin{itemize}
	\item подходит для обработки разных представлений тел,
	\item наиболее приближен к физическим моделям распространения света,
	\item позволяет реализовать оптические явления.
\end{itemize}

Недостатки:

\begin{itemize}
	\item большой объем работы,
	\item сложность вычислений.
\end{itemize}

За счет того, что алгоритм предоставляет возможность реализации поставленной задачи, за счет включения в алгоритм логики вычисления преломленных и отраженных лучей.


\subsection{Выбранный алгоритм}

Так как все описанные алгоритмы, кроме обратной трассировки,не учитывают искажений преломления света, выбран именно этот алгоритм.

\section{Освещение}

Модели освещения рассматривают 2 вида: локальная и глобальная.

\subsection{Модели освещения}

Локальная модель освещения учитывает модели индивидуально, что не подходит для реализации задачи, так как именно другие объекты будут видны в результате преломления. Что учитывает глобальная~\cite[с.~464,~с.~502,~с.~548]{rogers-book}.

Уравнение интенсивности:
\begin{equation}
	I = k_{\alpha} \cdot I_{\alpha} + k_d \cdot \sum_{i=1}^{N} (I_{L_i} \cdot \vec{n} \cdot \vec{L_i}) + k_s \cdot \sum_{i=1}^{N} (I_{L_i} \cdot (\vec{S} \cdot \vec{L_i}) ^ n) + k_s \cdot I_s + k_t \cdot I_t,
\end{equation}

где коэффициенты
\begin{itemize}
	\item $k_{\alpha}$ --- фоновый,
	\item $k_d$ --- диффузный,
	\item $k_s$ --- зеркальный,
	\item $k_t$ --- преломленного луча,
	\item $n$ --- аппроксимирующий распределение лучей от отражения,
\end{itemize}

и вектора
\begin{itemize}
	\item $\vec{L}$ --- направления до точечного источника света,
	\item $\vec{S}$ --- направление до наблюдателя,
	\item $\vec{n}$ --- нормали к поверхности.
\end{itemize}

\subsection{Тени}

Для определения интенсивности затененных областей удобно использовать алгоритм трассировки лучей, для это испускаются зондирующие лучи до источников света, если пересечение установлено, то область находится в тени~\cite[с.~517]{rogers-book}. Однако, когда рассматриваются прозрачные модели, точка может не быть затененной, поэтому для каждого зонда и источника коэффициент затенения от одного объекта составляет~\cite[с.~368]{cg-priciples}:

\begin{equation}
	k_{shadow,i} = k_{t, i}
\end{equation} 

Если луч пересекает несколько объектов, то коэффициент для $N_O$ полных пересечений моделей на отрезке до источника света будет:
\begin{equation}
	k_{shadow} = \prod_{i=1}^{N_O} k_{t, i}
\end{equation}

\subsection{Полная интенсивность}

Итоговая интенсивности в точке вычисляется по формуле~\ref{fn:intensity}.
\begin{equation} \label{fn:intensity}
		I = k_{\alpha} \cdot I_{\alpha} + 
		\sum_{i=1}^{N_L} \left[
			\prod_{j=1}^{N_O}	k_{t, j} \cdot 
			(k_d \cdot \vec{n} \cdot \vec{L_i} + k_s \cdot (\vec{S} \cdot \vec{L_i}) ^ n) 
		\right] \cdot I_{L_i} + 
		k_s \cdot I_s + 
		k_t \cdot I_t
\end{equation}

\section{Выводы из аналитической части}
В аналитической части достигнуты следующие цели:
\begin{itemize}
	\item формализована задача,
	\item рассмотрены представления объектов,
	\item проанализированы доступные алгоритмы удаления невидимых граней,
	\item описаны модели освещения.
\end{itemize}

В результате выбраны
\begin{enumerate}
	\item полигональное представление объектов, которое позволит изобразить модели любой формы;
	\item алгоритм обратной трассировки лучей, для поддержки преломления лучей света;
	\item глобальная модель освещения, так как она учитывает положения и освещенность других объектов, что и требуется для преломления света.
\end{enumerate}




\clearpage
